\documentclass[a4paper]{book}
\usepackage[times,inconsolata,hyper]{Rd}
\usepackage{makeidx}
\usepackage[utf8]{inputenc} % @SET ENCODING@
% \usepackage{graphicx} % @USE GRAPHICX@
\makeindex{}
\begin{document}
\chapter*{}
\begin{center}
{\textbf{\huge Package `FastrCAT'}}
\par\bigskip{\large \today}
\end{center}
\begin{description}
\raggedright{}
\inputencoding{utf8}
\item[Type]\AsIs{Package}
\item[Title]\AsIs{Analysis of Oceanographic Data from FastCATs}
\item[Version]\AsIs{0.0.0.9000}
\item[Author]\AsIs{person(given = ``Nissa'', family = ``Ferm'', email = ``nissa.ferm@noaa.gov'',
role = c(``aut'',``cre''))}
\item[Maintainer]\AsIs{Nissa Ferm }\email{nissa.ferm@noaa.gov}\AsIs{}
\item[Description]\AsIs{Complies data from .up files and converts them into a single dataframe.
Provides feedback to trouble shoot data quality issues from FastCat generated data. Produces
depth by temperature and salinity plots, maps of ...}
\item[Depends]\AsIs{R (>= 3.5.1)}
\item[License]\AsIs{CC0}
\item[Encoding]\AsIs{UTF-8}
\item[LazyData]\AsIs{true}
\item[Imports]\AsIs{dplyr (>= 0.7.6),
ggplot2 (>= 3.0.0),
grDevices (>= 3.5.1),
lubridate (>= 1.7.4),
magrittr (>= 1.5),
measurements (>= 1.2.0),
readr (>= 1.1.1),
stringr (>= 1.3.1),
tidyr (>= 0.8.1)}
\item[BugReports]\AsIs{}\url{https://gitlab.afsc.noaa.gov/Nissa.Ferm/FastrCAT/issues}\AsIs{}
\item[RoxygenNote]\AsIs{6.1.0}
\item[Suggests]\AsIs{knitr,
rmarkdown,
testthat,
htmltools}
\item[VignetteBuilder]\AsIs{knitr}
\end{description}
\Rdcontents{\R{} topics documented:}
\inputencoding{utf8}
\HeaderA{FastrCAT}{FastrCAT: A package for the Analysis of Oceanographic data from FastCATs.}{FastrCAT}
\aliasA{FastrCAT-package}{FastrCAT}{FastrCAT.Rdash.package}
%
\begin{Description}\relax
This package makes it easy to use temperature and salinity data collected
by a SeaBrid FastCAT. The package provides functions to read in .up files
and bind them into a single dataframe, provide QA/QC feedback, plot depth
by temperature/salinity for each cast, and maps of these... indicies.
\end{Description}
\inputencoding{utf8}
\HeaderA{make\_dataframe\_fc}{Create a Dataframe from .up files}{make.Rul.dataframe.Rul.fc}
%
\begin{Description}\relax
This function writes a single data frame in .csv format to file
containing oceanographic data collected by the FastCat during a cruise.
The format and column naming conventions are specific to the needs of EcoDAAT.
This is the primary function of the FastrCAT package and must be run prior to
all other functions. All other functions depend on the data frame generated.
Other outputs of the function are two text files. This first is a cruise
summary.The summary contains basic information about the data and some summary
statistics.The second is a warnings file, which tells users if data or
information is missing incase reprocessing is necessary.
\end{Description}
%
\begin{Usage}
\begin{verbatim}
make_dataframe_fc(current_path, GE = FALSE)
\end{verbatim}
\end{Usage}
%
\begin{Arguments}
\begin{ldescription}
\item[\code{current\_path}] The path to directory where all .up files
are located for a cruise.

\item[\code{GE}] A logical value, Returns the dataframe to the global R environment.
By defulaut it is set to FALSE. Set to TRUE if you would like the data
available in the global environment.
\end{ldescription}
\end{Arguments}
%
\begin{Value}
.csv file of all .up files. A .txt file of warnings and one
of summary statistics.
\end{Value}
\inputencoding{utf8}
\HeaderA{plot\_ts\_fc}{Temperature/Salinity by Depth Plots}{plot.Rul.ts.Rul.fc}
%
\begin{Description}\relax
Once make\_dataframe\_fc has been run, then plot\_ts\_fc
can be used. This function creates a depth by salinity and temperature
plot for each station. These are all .png files which will be located
in the plot folder within the current folder. It only needs to be run
once to generate a plot for each station. Each dot is a data point.
Check the profile for each station/haul.
\end{Description}
%
\begin{Usage}
\begin{verbatim}
plot_ts_fc(current_path)
\end{verbatim}
\end{Usage}
%
\begin{Arguments}
\begin{ldescription}
\item[\code{current\_path}] The path to directory where dataframe created from
make\_dataframe\_fc() is located.
\end{ldescription}
\end{Arguments}
%
\begin{Value}
A plot of temperature and salinity by depth for each
station of a cruise. Plots are written in the plot folder and
are in the .png format.
\end{Value}
\printindex{}
\end{document}
